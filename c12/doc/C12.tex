\documentclass{ctexart}
\usepackage[a4paper,left=1pt,right=1pt,bottom=1pt,top=1pt]{geometry}  % 设置边距
\usepackage{graphicx}  % 引入graphicx宏包
\usepackage{tikz}
\usepackage{multirow}
\usetikzlibrary{calc}
\usetikzlibrary {through,intersections,positioning,arrows.meta}
\usepackage[table]{xcolor}
\usetikzlibrary{patterns,snakes}
\usetikzlibrary{matrix}
\pgfdeclarelayer{background}
\pgfdeclarelayer{foreground}
\pgfsetlayers{background,main,foreground}
\usepackage{eqparbox}
\usepackage{longtable, array}
\usepackage{multirow}
\usepackage{minted}
\usepackage{booktabs}
\usepackage{float,afterpage}
\usepackage{bytefield}

\newbox\matrixcellbox

\newcommand{\bitrect}[2]{
  \begin{pgfonlayer}{foreground}
    \pgfmathsetmacro\cellwidth{#1*0.6}
    \draw [thick] (0,0) rectangle (\cellwidth,1.5) coordinate[midway,below=15pt] (b1);
    \pgfmathsetmacro\result{#1-1}
    \foreach \x in {1,...,\result}
      \draw [thick] (\x*0.6,1.5) -- (\x*0.6, 1.3);
  \end{pgfonlayer}
%  \node [below left, align=right] at (0,0) {Type \\ Reset};
  \bitlabels{#1}{#2}
}
\newcommand{\rwbits}[3]{
  \draw [thick] (#1*0.6,0) rectangle ++(#2*0.6,1.5) node[align=center,pos=0.5]{#3} ;
%   \pgfmathsetmacro\start{#1+0.5}
%   \pgfmathsetmacro\finish{#1+#2-0.5}
%  \foreach \x in {\start,...,\finish}
%    \node [below, align=center] at (\x, 0) {R/W \\ 0};
}
\newcommand{\robits}[3]{
  \begin{pgfonlayer}{background}
    \draw [thick, fill=lightgray] (#1*0.6,0) rectangle ++(#2*0.6,1) node[rotate=90,pos=0.5]{#3};
  \end{pgfonlayer}
  % \pgfmathsetmacro\start{#1+0.5}
  % \pgfmathsetmacro\finish{#1+#2-0.5}
%  \foreach \x in {\start,...,\finish}
%    \node [below, align=center] at (\x, 0) {RO \\ 0};
}
\newcommand{\bitlabels}[2]{
  \foreach \bit in {1,...,#1}{
     \pgfmathsetmacro\result{#2}
     \node [above] at (\bit*0.6-0.3, 1.5) {\pgfmathprintnumber{\result}} ;
   }
}

\newcommand{\memsection}[4]{%
    % define the height of the memsection
    \bytefieldsetup{bitheight=#3\baselineskip}%
    \bitbox[]{10}{%
        \texttt{#1}% print end address
        \\
        % do some spacing
        \vspace{#3\baselineskip}
        \vspace{-2\baselineskip}
        \vspace{-#3pt}
        \texttt{#2}% print start address
    }%
    \bitbox{16}[bgcolor=white]{#4}% print box with caption
}

\tikzset{
    l column/.style={
        execute at begin 
        node={\setbox\matrixcellbox=\hbox\bgroup},
        execute at end
        node={\egroup\eqmakebox[\tikzmatrixname\the\pgfmatrixcurrentcolumn][l]{\copy\matrixcellbox}}
    },
    register/.style={matrix of nodes,nodes={l column},draw,rounded corners,column sep=0pt,
        execute at end matrix={
            \draw (\tikzmatrixname-1-1.south) -- (\tikzmatrixname-2-1.north)
            coordinate[midway](tmpH)
            (\tikzmatrixname-1-\the\numexpr\pgfmatrixcurrentcolumn-1\relax.east) -- (\tikzmatrixname-1-\the\pgfmatrixcurrentcolumn.west) 
            coordinate[midway](tmpV);
            \draw (\tikzmatrixname.west|-tmpH) -- (\tikzmatrixname.east|-tmpH) 
            (\tikzmatrixname.north-|tmpV) -- (\tikzmatrixname.south-|tmpV);
        }
    },
    do path picture/.style={%
      path picture={%
        \pgfpointdiff{\pgfpointanchor{path picture bounding box}{south west}}%
          {\pgfpointanchor{path picture bounding box}{north east}}%
        \pgfgetlastxy\x\y%
        \tikzset{x=\x/2,y=\y/2}%
        #1
      }
    },
    plus/.style={do path picture={    
      \draw [line cap=round] (-3/4,0) -- (3/4,0) (0,-3/4) -- (0,3/4);
    }}
}

\begin{document}

  \begin{figure}[htpb]
    \begin{tikzpicture}
    
      \bitrect{32}{32-\bit}
      \rwbits{0}{8}{段基地址31 \textasciitilde 24}
      \rwbits{8}{1}{G}
      \rwbits{9}{1}{D \\ / \\ B}
      \rwbits{10}{1}{L}
      \rwbits{11}{1}{A \\ V \\ L}
      \rwbits{12}{4}{段界限 \\ 19\textasciitilde 16}
      \rwbits{16}{1}{P}
      \rwbits{17}{2}{DPL}
      \rwbits{19}{1}{S}
      \rwbits{20}{4}{TYPE}
      \rwbits{24}{8}{段基地址23\textasciitilde 16}
      \coordinate (G) at (8*0.6+0.3,0);
      \matrix[register,below=2cm of G.south,xshift=-3cm] (GDesc){
        G & 段界限粒度位\\
        0  & 字节为单位\\
        1  & 4KB为单位\\
      };
      \draw[-stealth] (G) -- (8*0.6+0.3,-1) -| (GDesc);
      %
      \coordinate (D) at (9*0.6+0.3,0);
      \matrix[register,below=2cm of D.south] (DDesc){
        D/B & 默认操作尺寸\\
        0  & 16位\\
        1  & 32位\\
      };
      \draw[-stealth] (D) -- (9*0.6+0.3,-1) -| (DDesc);
      %
      \coordinate (L) at (10*0.6+0.3,0);
      \matrix[register,below=6cm of L.south,xshift=1.5cm] (LDesc){
        L & 64位代码段标志\\
        0  & 32位\\
        1  & 64位\\
      };
      \draw[-stealth] (L) -- (10*0.6+0.3,-1) -| (LDesc);
      %
      \coordinate (AVL) at (11*0.6+0.3,0);
      \node [draw,below=1cm of AVL.south,xshift=2cm,align=center] (AVLDesc) {软 \\ 件 \\ 可 \\使 \\用 \\的\\位\\置};
      \draw[-stealth] (AVL) -- (11*0.6+0.3,-0.5) -| (AVLDesc);
      %
      \coordinate (P) at (16*0.6+0.3,0);
      \node [draw,below=1cm of P.south,align=center] (PDesc) {段 \\ 存 \\ 在 \\位};
      \draw[-stealth] (P) -- (16*0.6+0.3,-0.5) -| (PDesc);
      %
      \coordinate (DPL) at (17*0.6+0.3,0);
      \matrix[register,below=6cm of L.south,xshift=6cm] (DPLDesc){
        DPL & 描述符特权级\\
        0  & 最高特权级\\
        1  & \\
        2  & \\
        3  & 最低特权级\\
      };
      \draw[-stealth] (DPL) -- (17*0.6+0.3,-5) -| (DPLDesc);
      %
      \coordinate (S) at (19*0.6+0.3,0);
      \matrix[register,below=2cm of S.south,xshift=1cm] (SDesc){
        S & 描述符的类型\\
        0  & 系统段\\
        1  & 代码段或数据段\\
      };
      \draw[-stealth] (S) -- (19*0.6+0.3,-1) -| (SDesc);
      %
      \coordinate (TYPE) at (22*0.6+0.3,0);
      \node [draw,below=1cm of TYPE.south,align=center,xshift=2cm] (TYPEDesc) {参考 表\ref{demo-table}};
      \draw[-stealth] (TYPE) -- (22*0.6+0.3,-0.5) -| (TYPEDesc);

   
    \end{tikzpicture}
    \begin{tikzpicture}
      \bitrect{32}{32-\bit}
      \rwbits{0}{16}{段基地址15 \textasciitilde 0}
      \rwbits{16}{16}{段界限15 \textasciitilde 0}
    \end{tikzpicture}
    \caption{段描述符}

  \end{figure}
  %
  \begin{table}[htpb]
    \begin{center}
    
      \begin{tabular}{|c | c | c | c| c|c|} 
        \hline
        \rowcolor{gray!30}
        X &  E &  W & A & 描述符类别 & 含义 \\ \hline
        \textcolor{red}{0} &  0 &  0 & X & \multirow{4}{*}{数据} & 只读\\ \cline{1-4}  \cline{6-6}
        \textcolor{red}{0} &  0 &  1 & X &  & 读、写\\ \cline{1-4}  \cline{6-6}
        \textcolor{red}{0} &  1 &  0 & X &  & 只读、向下扩展\\ \cline{1-4}  \cline{6-6}
        \textcolor{red}{0} &  1 &  1 & X & & 读、写、向下扩展\\ \hline
        \rowcolor{gray!30}
        X & C & R & A & 描述符类别 & 含义 \\ \hline
        \textcolor{blue}{1} & 0 & 0 & X & \multirow{4}{*}{代码} & 只执行 \\ \cline{1-4}  \cline{6-6}
        \textcolor{blue}{1} & 0 & 1 & X & & 只执行 \\ \cline{1-4}  \cline{6-6}
        \textcolor{blue}{1} & 1 & 0 & X & & 只执行、依从的代码段 \\ \cline{1-4}  \cline{6-6}
        \textcolor{blue}{1} & 1 & 1 & X & & 执行、读、依从的代码段 \\ \hline
      \end{tabular}
      \caption{代码段和数据段描述符的TYPE字段}
    \label{demo-table}
    \end{center}
  \end{table}

  % GDT和GDTR的关系
  \begin{figure}[h]
    \centering
    \begin{tikzpicture}
      % 画一个方框并在其中添加文字
      \node[draw, minimum width=100, minimum height=100, align=center] at (0,0) {处理器};
      \node[draw, minimum width=60, minimum height=20, align=center] at (0.5,1) {GDTR};
      \node[draw, shape=circle,inner sep=0.5mm,fill=red!20] at (1.2, 1) (A) {};
      \node[draw, minimum width=100, minimum height=30, align=center,right =5cm of A,anchor=north west] (MB) {};
      \node[draw, minimum width=100, minimum height=20, align=center,above =0 of MB] (GDT) {GDT};
      \node[draw, minimum width=100, minimum height=40, above =0 of GDT] (MU) {};
      \node[draw, minimum width=100, minimum height=150, above =0 of MU] (E) {};
      \draw[->,>={Stealth}] (A) -- (MB.north west);
      \node[left = 0.5 of MB,anchor=north east,] (F) {00000000};
      \node[left = 0.5 of MU,anchor=south east,yshift=10] (G) {000FFFFF};
      \node[left = 0.5 of E,anchor=south east,yshift=70] (H) {FFFFFFFF};
    \end{tikzpicture}
    \caption{GDT和GDTR的关系}
    \label{fig:GDTGDTR}
  \end{figure}

  \newpage
  % 全局描述符表寄存器GDTR
  \begin{figure}[h]
    \centering
    \begin{bytefield}[bitheight=\widthof{~Sign~},
      boxformatting={\centering}]{48}
      \bitheader[endianness=big]{47,16,15,0} \\
      \bitbox{32}{全局描述符表线性基地址} &
      \bitbox{16}{全局描述符表边界} 

      \end{bytefield}
      \caption{全局描述符表寄存器GDTR}
      \label{fig:GDTR}
  \end{figure}

  % \newpage
  %段寄存器与段选择子
  \begin{figure}[h]
   
    \centering
    \begin{tikzpicture}[>=latex]   
        

            % \foreach \x in {0,...,15}
            %   \node [right=0.5*\x of Selector.north west] (a\x) {\x} ;
            \node at (10,0) [rectangle,draw=black,thick,minimum width=50,anchor=west] (CS)  {CS};
            \node [rectangle,draw=black,dashed,minimum width=150,right=of CS] (CSCache) {段的线性基地址、界限和属性};
            \node [rectangle,draw=black,thick,minimum width=50,below=0.5 of CS] (DS) {DS};
            \node [rectangle,draw=black,dashed,minimum width=150,right=of DS] (DSCache) {段的线性基地址、界限和属性};
            \node [rectangle,draw=black,thick,minimum width=50,below=0.5 of DS] (ES) {ES};
            \node [rectangle,draw=black,dashed,minimum width=150,right=of ES] (ESCache) {段的线性基地址、界限和属性};
            \node [rectangle,draw=black,thick,minimum width=50,below=0.5 of ES] (FS) {FS};
            \node [rectangle,draw=black,dashed,minimum width=150,right=of FS] (FSCache) {段的线性基地址、界限和属性};
            \node [rectangle,draw=black,thick,minimum width=50,below=0.5 of FS] (GS) {GS};
            \node [rectangle,draw=black,dashed,minimum width=150,right=of GS] (GSCache) {段的线性基地址、界限和属性};
            \node [rectangle,draw=black,thick,minimum width=50,below=0.5 of GS] (SS) {SS};
            \node [rectangle,draw=black,dashed,minimum width=150,right=of SS] (SSCache) {段的线性基地址、界限和属性};

            % \node at (0,-4)  [rectangle,draw=black,thick,inner sep=10,minimum width=4] (Selector) {Selector};
            \draw [thick] (0,-3) rectangle (6.4,-2) coordinate[below=0.5] (Selector) node[ pos=0.5, anchor=south,above=1] {\textcolor{red}{段选择子  (Selector)}};
            \foreach \x in {1,...,16}
              \draw [thick] (\x*0.4,-2) -- (\x*0.4, -2.1);
            \foreach \bit in {0,...,15} 
              \node [above] at (6.4 - \bit*0.4-0.2, -2) {\bit};
            \draw [thick] (0,-3) rectangle (5.2,-2) node[pos=0.5]{描述符索引};
            \draw [thick] (5.2,-3) rectangle (5.6,-2) coordinate[midway,below=0.5] (TI) node[pos=0.5,align=center] {T \\ I};
            \draw [thick] (5.6,-3) rectangle (6.4,-2) coordinate[midway,below=0.5] (RPL) node[pos=0.5,align=center] {RPL};  


            \matrix[register,below=1 of TI.south,xshift=-120] (TIDesc){
              TI & 描述符表指示器\\
              0  & 描述符在GDT中\\
              1  & 描述符在LDT中 \\
            };

            \node [draw,below=1 of RPL.south,align=left] (RPLDesc) {请求特权级:\\
            给出当前选择子的\\那个程序的特权级别\\正是该程序要求访问这个内存段};


            % \draw[-stealth] (TI.south) -- (TIDesc);
            % \node [above] at ($(Selector.north west)!0.5!(a1)$) {15} ;
            % \node [above] at ($(a2)!0.5!(a3)$) {14} ;
            % \node [below=15pt of Selector]  {aa} ;
            \draw [->] (Selector.east) -- (CS.west);
            \draw [->] (Selector.east) -- (DS.west);
            \draw [->] (Selector.east) -- (ES.west);
            \draw [->] (Selector.east) -- (FS.west);
            \draw [->] (Selector.east) -- (GS.west);
            \draw [->] (Selector.east) -- (SS.west);
            % \node[] (n1) [below=of SS.west] {} ;

            % \node[] (n2) [below= of n1] {} ;

            % \node[] (n3) [below=of SS.east] {} ;

            % \node[] (n4) [below= of n3] {} ;

            % \draw (n1) -- (n2) ;
            % \draw (n3) -- (n4) ;

            % \node[] (a) at ($(n1)!0.5!(n2)$) {};
            % \node[] (b) at ($(n3)!0.5!(n4)$) {};

            \draw [
              thick,
              decoration={
                  brace,
                  mirror,
                  raise=0.5cm
              },
              decorate
            ] (SS.south west) -- (SS.south east)  
            node [pos=0.5,anchor=north,yshift=-20] {\textcolor{red}{段选择器}}; 

            \draw [
              thick,
              decoration={
                  brace,
                  mirror,
                  raise=0.5cm
              },
              decorate
            ] (SSCache.south west) -- (SSCache.south east)  
            node [pos=0.5,anchor=north,yshift=-20] {\textcolor{red}{描述符高速缓存器}}; 
            % \draw[<->](a.center) --node[fill=white]{段选择器} (b.center) ;

            % \begin{pgfonlayer}{foreground}
            %   \pgfmathsetmacro\cellwidth{16*0.6}
            %   \draw [thick,left=5cm of ES] (0,0) rectangle (\cellwidth,1.5) coordinate[midway,below=15pt] (b1);
            %   \pgfmathsetmacro\result{15}
            %   \foreach \x in {1,...,\result}
            %     \draw [thick] (\x*0.6,1.5) -- (\x*0.6, 1.3);
            % \end{pgfonlayer}

            % \foreach \x in {1,...,15}
            % \draw [thick] (\x*0.5,1.5) -- (\x*0.5, 1.3);
     
    \end{tikzpicture}

    \caption{段寄存器与段选择子}
    \label{fig:example}
  \end{figure}
  \newpage
  % 段选择器和描述符高速缓存器的加载过程
  \begin{figure}[h]
    \centering
    \begin{tikzpicture}
      \node[draw] at (-4,-1 ) (selector) {段选择子(TI=0)} ;
      \node[draw,right =1 of selector] (DS) {DS} ;
      \draw[thick, red, ->] (selector) -- node[above]{1}(DS);
      \node[draw,dashed,right =0.1 of DS,minimum width=40, minimum height=15] (empty) {} ;
      \coordinate[right = 10 of empty] (pivot1) {} ;
      % 画一个方框并在其中添加文字
      \node[draw, minimum width=100, minimum height=100, align=center] at (0,0) {处理器};
      \node[draw, minimum width=50, minimum height=20, align=center] at (0.5,1) {GDTR};
      \node[draw, shape=circle,inner sep=0.5mm,fill=red!20] at (1.2, 1) (A) {};
      \node [circle, draw,plus,scale=2] at (3,1) (circleplus)   {};
  
      \node[draw, minimum width=100, minimum height=20, align=center,right =5cm of A,anchor=north west,yshift=-20] (MB) {MB};
      \node[draw, minimum width=100, minimum height=40, align=center,above =0 of MB] (GDT) {GDT};
      \coordinate[above = 2 of GDT.south west] (MU) {};
      \node[draw, minimum width=100, minimum height=150, above =0 of GDT] (E) {};
      \draw[->,>={Stealth}] (A) -- (circleplus);
      \draw[->,>={Stealth},red] (circleplus) --node[above]{2} (GDT);
      % \draw[->,>={Stealth}] (A)  --node[below]{dd} (MB.north west);
      \node[left = 0.5 of MB,anchor=north east,] (F) {00000000};
    
      \node[left = 0.5 of E,anchor=south east,yshift=70] (H) {FFFFFFFF};
      \coordinate[above = 2 of circleplus] (B) {} ;
      \draw[thick, red, ->] (DS.north) |- node[above,xshift=20]{X8}(B) -- (circleplus);
  
      \node[draw, shape=circle,inner sep=0.5mm,fill=red!20,above=1 of MB.south,xshift=10] (C) {};
      \draw[thick, red, ->] (GDT) -| node[above]{3}(pivot1) -- (empty);
    \end{tikzpicture}
    \caption{段选择器和描述符高速缓存器的加载过程}
    \label{fig:cacheload}
  \end{figure}

  % 保护模式下的内存访问
  \begin{figure}[h]
    \centering
    \begin{tikzpicture}
      \node (code) at (0,6) {mov byte[0x00],'p'};
      \draw[thick] ([xshift=-5mm]code.south) -- ([xshift=5mm]code.south);

      \node[draw] at (-4,-1 ) (selector) {段选择子} ;
      \node[draw,right =2 of selector] (DS) {DS} ;
      \node[draw,dashed,right =0.1 of DS,minimum width=40, minimum height=15] (empty) {} ;
      % 画一个方框并在其中添加文字
      \node[draw, minimum width=100, minimum height=100, align=center] (cpu) at (0,0) {处理器};
      % \node[draw, minimum width=50, minimum height=20, align=center] at (0.5,1) {GDTR};
      % \node[draw, shape=circle,inner sep=0.5mm,fill=red!20] at (1.2, 1) (A) {};
      \node [circle, draw,plus,scale=2] at (3,3) (circleplus)   {};
  
      \node[draw, minimum width=100, minimum height=145, align=center,right =5cm of cpu] (MB) {MB};
      \node[draw, minimum width=100, minimum height=20, align=center,above =0 of MB] (P) {P};
      \coordinate[above = 2 of P.south west] (MU) {};
      \node[draw, minimum width=100, minimum height=150, above =0 of P] (E) {};
      % \draw[->,>={Stealth}] (A) -- (circleplus);
      \draw[->,>={Stealth}] (circleplus) --node[above]{000B8000} (P);
      % \draw[->,>={Stealth}] (A)  --node[below]{dd} (MB.north west);
      \node[left = 0.5 of MB,anchor=north east,yshift=-60] (F) {00000000};
    
      \node[left = 0.5 of E,anchor=south east,yshift=70] (H) {FFFFFFFF};
      \coordinate[above = 2 of circleplus] (B) {} ;
      % \draw[thick, red, ->] (DS.north) |- node[above,xshift=20]{X8}(B) -- (circleplus);
  
      % \node[draw, shape=circle,inner sep=0.5mm,fill=red!20,above=1 of MB.south,xshift=10] (C) {};

      \draw[thick, red, ->] (empty) -| node[above,fill=white]{32位基地址}(circleplus) ;
      \draw[thick, red, ->] (code.south) |- node[midway,fill=white,yshift=20]{段内偏移}(circleplus) ;
    \end{tikzpicture}
    \caption{保护模式下的内存访问}
    \label{fig:memoryAccess}
  \end{figure}
  \newpage
  % 保护模式下的处理器取指令的过程
  \begin{figure}[h]
  \centering
  \begin{tikzpicture}
    % \node (code) at (0,6) {mov byte[0x00],'p'};
    % \draw[thick] ([xshift=-5mm]code.south) -- ([xshift=5mm]code.south);

    % \node[draw] at (-4,-1 ) (selector) {段选择子} ;
    \node[draw,right =2 of selector] (DS) {CS} ;
    \node[draw,dashed,right =0.1 of DS,minimum width=40, minimum height=15] (empty) {} ;
    % 画一个方框并在其中添加文字
    \node[draw, minimum width=100, minimum height=100, align=center] (cpu) at (0,0) {处理器};
    \node[draw, minimum width=50, minimum height=20, align=center] at (0.5,1) {EIP};
    \node[draw, shape=circle,inner sep=0.5mm,fill=red!20] at (1.2, 1) (A) {};
    \node [circle, draw,plus,scale=2] at (3,1) (circleplus)   {};

    \node[draw, minimum width=100, minimum height=40, align=center,right =5cm of cpu] (MB) {MB};
    \node[draw, minimum width=100, minimum height=20, align=center,above =0 of MB] (P) {指令};
    \coordinate[above = 2 of P.south west] (MU) {};
    \node[draw, minimum width=100, minimum height=150, above =0 of P] (E) {};
    \draw[->,>={Stealth}] (A) -- (circleplus);
    \draw[->,>={Stealth}] (circleplus) -- (P);
    % \draw[->,>={Stealth}] (A)  --node[below]{dd} (MB.north west);
    \node[left = 0.5 of MB,anchor=north east,yshift=-10] (F) {00000000};
  
    \node[left = 0.5 of E,anchor=south east,yshift=60] (H) {FFFFFFFF};
    \coordinate[above = 2 of circleplus] (B) {} ;
    % \draw[thick, red, ->] (DS.north) |- node[above,xshift=20]{X8}(B) -- (circleplus);

    % \node[draw, shape=circle,inner sep=0.5mm,fill=red!20,above=1 of MB.south,xshift=10] (C) {};

    \draw[thick, red, ->] (empty) -| node[above,fill=white]{32位基地址}(circleplus) ;
    % \draw[thick, red, ->] (code.south) |- node[midway,fill=white,yshift=20]{段内偏移}(circleplus) ;
  \end{tikzpicture} 
  \caption{保护模式下处理器取指令的过程}
  \label{fig:instructionAccess}
\end{figure}


  \begin{longtable}{ |*{1}{m{.55\paperwidth}} | *{1}{m{.40\paperwidth}} |}
      \toprule
      \begin{minted}[escapeinside=??]{nasm}
.text
entry start
start:
mov ax,cs
mov ss,ax
mov sp,#0x7c00  
      \end{minted}
      & 初始化栈,参考保护模式前的内存映像图\ref{fig:memoryMap}  \\
      \cline{2-2}
      \begin{minted}[escapeinside=??]{nasm}
mov ah,#0x00
mov al,#0x03
int 0x10
      \end{minted} 
      &   bios清屏功能  \\
      \cline{2-2}
      \begin{minted}[escapeinside=??]{nasm}
seg cs
mov ax, gdt_base+0x7c00 
seg cs
mov dx, gdt_base+0x7c00 + 0x02
mov bx,#16
div bx  
      \end{minted} 
      & 
      \colorbox{gray!80} {\Large div bx}
      $$
          \frac{DX:AX}{BX} = AX.DX
      $$ 
      商在 AX 中, 余数在 DX 中  \newline 
      在此示例中: \newline
      将GDT的基地址右移4位得到段地址存到AX中,段内偏移地址存到dx中    \\
      \cline{2-2}
      \begin{minted}[escapeinside=??]{nasm}
mov ds,ax
mov bx,dx
      \end{minted} 
      & 将gdt基地址的段地址存到ds,偏移地址存到bx中  \\
      \cline{2-2}
      \begin{minted}[escapeinside=??]{nasm}
mov dword 0(bx),#0x00
mov dword 4(bx),#0x00
      \end{minted}     
      & GDT的第一个描述符为空描述符   \\
      \cline{2-2}
      \begin{minted}[escapeinside=??]{nasm}
mov dword 0x08(bx),#0x8000ffff
mov dword 0x0c(bx),#0x0040920b
      \end{minted} 
      & 
      段线性地址=0x000B8000   \newline
      G=0:段粒度为字节     \newline
      段界限=FFFF:结合G=0,该段的长度为64KB   \newline
      S=1:存储器的段          \newline
      D=1:32位操作尺寸       \newline
      P=1:该段目前位于内存中   \newline
      DPL=000:段的特权级为0    \newline
      TYPE=0010:可读可写、向上扩展的数据段    \\
      \cline{2-2}
      \begin{minted}[escapeinside=??]{nasm}
seg cs
mov word gdt_size + 0x7c00,#15
      \end{minted} 
      & 设置gdt的边界     \\
      \cline{2-2}
      \begin{minted}[escapeinside=??]{nasm}
seg cs
lgdt gdt_size+0x7c00
      \end{minted}   
      &       
      \colorbox{gray!80}{\Large ldgt m}:加载描述符表的线性基地址和界限   \newline
      在有效地址m处,包含了GDT的32位线性地址和16位界限值,共6字节,参考图\ref{fig:GDTR}  \\
      \cline{2-2}
      \begin{minted}[escapeinside=??]{nasm}
in al,#0x92 
or al,#0x02
out 0x92,al 
cli
      \end{minted}   
      & 打开A20 \\
      \cline{2-2}
      \begin{minted}[escapeinside=??]{nasm}
mov eax,cr0
or eax,#1
mov cr0,eax 
      \end{minted}   
      & 开启保护模式 \\
      \cline{2-2}
      \begin{minted}[escapeinside=??]{nasm}
mov cx,#0x08
mov ds,cx
      \end{minted}   
      & 将段选择子传到段选择器,参考图\ref{fig:cacheload} \\
      \cline{2-2}
      \begin{minted}[escapeinside=??]{nasm}
mov byte 0x00,#0x50
mov byte 0x02,#0x72
mov byte 0x04,#0x6F
mov byte 0x06,#0x74
mov byte 0x08,#0x65
mov byte 0x0a,#0x63
mov byte 0x0c,#0x74
mov byte 0x0e,#0x20
mov byte 0x10,#0x6D
mov byte 0x12,#0x6F
mov byte 0x14,#0x64
mov byte 0x16,#0x65
mov byte 0x18,#0x20
mov byte 0x1A,#0x4F
mov byte 0x01C,#0x4B
mov byte 0x1E,#0x2E
      \end{minted}    
      & 将数据传到到段描述符指定的线性地址加偏移量处 \\
      \cline{2-2}
      \begin{minted}[escapeinside=??]{nasm}
hlt
      \end{minted}   
      & cpu暂停  \\
      \cline{2-2}
      \begin{minted}[escapeinside=??]{nasm}
gdt_size:         .word 0
      \end{minted} 
      & 存放GDT的界限值 \\
      \cline{2-2}
      \begin{minted}[escapeinside=??]{nasm}
gdt_base:   .word 0x7e00,0x0000
      \end{minted}    
      & 声明GDT的起始物理地址 \\
      \cline{2-2}
      \begin{minted}[escapeinside=??]{nasm}
.org 510
.word 0xAA55
      \end{minted}   
      & cpu暂停  \\
      \cline{2-2}
  \end{longtable}
 
  \begin{figure}
    \centering
    \begin{bytefield}{26}
      \memsection{FFFF FFFF}{}{15}{-- free --}\\
      \begin{rightwordgroup}{1MB}
        \memsection{000F FFFF}{}{4}{} \\
        \memsection{0001 7DFF}{0000 7E00}{4}{GDT,最大64KB}\\
        \memsection{}{0000 7C00}{3}{主引导程序,512字节}\\
        \memsection{}{0000 0000}{8}{栈空间,从7c00向下推进}
      \end{rightwordgroup}\\
    \end{bytefield}
    \caption{进入保护模式前的内存映像}
    \label{fig:memoryMap}
  \end{figure}

  
\end{document}